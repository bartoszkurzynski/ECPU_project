% Options for packages loaded elsewhere
\PassOptionsToPackage{unicode}{hyperref}
\PassOptionsToPackage{hyphens}{url}
%
\documentclass[
]{article}
\usepackage{amsmath,amssymb}
\usepackage{iftex}
\ifPDFTeX
  \usepackage[T1]{fontenc}
  \usepackage[utf8]{inputenc}
  \usepackage{textcomp} % provide euro and other symbols
\else % if luatex or xetex
  \usepackage{unicode-math} % this also loads fontspec
  \defaultfontfeatures{Scale=MatchLowercase}
  \defaultfontfeatures[\rmfamily]{Ligatures=TeX,Scale=1}
\fi
\usepackage{lmodern}
\ifPDFTeX\else
  % xetex/luatex font selection
\fi
% Use upquote if available, for straight quotes in verbatim environments
\IfFileExists{upquote.sty}{\usepackage{upquote}}{}
\IfFileExists{microtype.sty}{% use microtype if available
  \usepackage[]{microtype}
  \UseMicrotypeSet[protrusion]{basicmath} % disable protrusion for tt fonts
}{}
\makeatletter
\@ifundefined{KOMAClassName}{% if non-KOMA class
  \IfFileExists{parskip.sty}{%
    \usepackage{parskip}
  }{% else
    \setlength{\parindent}{0pt}
    \setlength{\parskip}{6pt plus 2pt minus 1pt}}
}{% if KOMA class
  \KOMAoptions{parskip=half}}
\makeatother
\usepackage{xcolor}
\usepackage[margin=1in]{geometry}
\usepackage{graphicx}
\makeatletter
\def\maxwidth{\ifdim\Gin@nat@width>\linewidth\linewidth\else\Gin@nat@width\fi}
\def\maxheight{\ifdim\Gin@nat@height>\textheight\textheight\else\Gin@nat@height\fi}
\makeatother
% Scale images if necessary, so that they will not overflow the page
% margins by default, and it is still possible to overwrite the defaults
% using explicit options in \includegraphics[width, height, ...]{}
\setkeys{Gin}{width=\maxwidth,height=\maxheight,keepaspectratio}
% Set default figure placement to htbp
\makeatletter
\def\fps@figure{htbp}
\makeatother
\setlength{\emergencystretch}{3em} % prevent overfull lines
\providecommand{\tightlist}{%
  \setlength{\itemsep}{0pt}\setlength{\parskip}{0pt}}
\setcounter{secnumdepth}{-\maxdimen} % remove section numbering
\usepackage{booktabs}
\usepackage{longtable}
\usepackage{array}
\usepackage{multirow}
\usepackage{wrapfig}
\usepackage{float}
\usepackage{colortbl}
\usepackage{pdflscape}
\usepackage{tabu}
\usepackage{threeparttable}
\usepackage{threeparttablex}
\usepackage[normalem]{ulem}
\usepackage{makecell}
\usepackage{xcolor}
\ifLuaTeX
  \usepackage{selnolig}  % disable illegal ligatures
\fi
\usepackage{bookmark}
\IfFileExists{xurl.sty}{\usepackage{xurl}}{} % add URL line breaks if available
\urlstyle{same}
\hypersetup{
  pdftitle={brudnopis},
  hidelinks,
  pdfcreator={LaTeX via pandoc}}

\title{brudnopis}
\author{}
\date{\vspace{-2.5em}2025-01-31}

\begin{document}
\maketitle

\section{Analiza czynników wpływających na wyniki z egzaminów
studentów}\label{analiza-czynnikuxf3w-wpux142ywajux105cych-na-wyniki-z-egzaminuxf3w-studentuxf3w}

\subsection{Wstęp}\label{wstux119p}

\subsubsection{Cel projektu}\label{cel-projektu}

Celem projektu jest analiza czynników wpływających na wyniki uczniów w
egzaminach. Badanie ma na celu zidentyfikowanie, które zmienne, takie
jak liczba godzin nauki, frekwencja czy wsparcie rodziny, mają
największy wpływ na sukces akademicki. Wyniki analizy mogą pomóc w
optymalizacji strategii edukacyjnych oraz wdrożeniu skutecznych metod
wsparcia dla uczniów o niższych wynikach.

\subsubsection{Opis problemu}\label{opis-problemu}

Osiągnięcia akademickie uczniów są determinowane przez wiele czynników
-- zarówno indywidualnych, jak i środowiskowych. Trudności w nauce,
poziom zaangażowania rodziców, dostęp do zasobów edukacyjnych czy
aktywności pozalekcyjne mogą znacząco wpływać na ostateczne wyniki.
Analiza tych czynników pozwoli na lepsze zrozumienie zależności i
wsparcie uczniów w poprawie wyników.

\subsubsection{Znaczenie analizy wyników
uczniów}\label{znaczenie-analizy-wynikuxf3w-uczniuxf3w}

Zrozumienie czynników wpływających na wyniki egzaminacyjne jest istotne
z kilku powodów:

Optymalizacja strategii nauczania -- analiza wyników może pomóc
nauczycielom i decydentom edukacyjnym w dostosowaniu metod nauczania do
potrzeb uczniów.

Wsparcie dla uczniów w trudnej sytuacji -- identyfikacja kluczowych
czynników umożliwia wdrożenie skutecznych programów pomocy.

Poprawa efektywności edukacji -- lepsze zrozumienie procesów nauczania
pozwala na skuteczniejsze zarządzanie systemem edukacyjnym.

\subsection{Opis danych}\label{opis-danych}

\subsubsection{Źródło danych}\label{ux17aruxf3dux142o-danych}

Dane wykorzystane w analizie pochodzą z anonimowego zbioru danych
dotyczącego uczniów szkół średnich. Zawierają one informacje na temat
wyników egzaminacyjnych oraz czynników potencjalnie wpływających na
sukces akademicki.

\subsubsection{Struktura zbioru danych}\label{struktura-zbioru-danych}

Zbiór danych zawiera następujące kolumny:

Zmienne numeryczne: Liczba godzin nauki, frekwencja, liczba sesji
korepetycji, wynik egzaminu itp.

Zmienne kategoryczne: Poziom zaangażowania rodziców, dostęp do zasobów
edukacyjnych, typ szkoły, aktywności pozalekcyjne itp.

\subsubsection{Opis zmiennych}\label{opis-zmiennych}

Każda z analizowanych zmiennych pełni określoną funkcję w kontekście
analizy wyników uczniów. Główne zmienne to:

\begin{itemize}
\item
  Hours Studied -- liczba godzin nauki tygodniowo.
\item
  Attendance -- procentowa frekwencja na zajęciach.
\item
  Tutoring Sessions -- liczba godzin spędzonych na korepetycjach.
\item
  Parental Involvement -- poziom zaangażowania rodziców (Low, Medium,
  High).
\item
  Access to Resources -- dostęp do zasobów edukacyjnych (Low, Medium,
  High).
\item
  Motivation Level -- poziom motywacji ucznia.
\item
  Previous Scores -- wyniki ucznia z poprzednich egzaminów.
\item
  Exam Score -- wynik końcowy z egzaminu.
\end{itemize}

Dane te zostaną poddane dalszej analizie w celu określenia ich wpływu na
końcowe wyniki uczniów.

\subsection{Przygotowanie danych}\label{przygotowanie-danych}

\subsubsection{Wizualizacja brakujących
danych}\label{wizualizacja-brakujux105cych-danych}

Aby lepiej zrozumieć rozkład brakujących danych w naszym zbiorze,
wykonujemy kilka analiz:

\begin{itemize}
\tightlist
\item
  Wykres \texttt{missing\_plot()} przedstawia ogólny wzór brakujących
  wartości dla kluczowych zmiennych.
\end{itemize}

Po zwizualizowaniu brakujących wartości za pomocą wykresu
\texttt{missing\_plot()} można zauważyć, że w zbiorze danych występują
braki w kilku zmiennych.

Po przealizowaniu przedstawionych wykresów można dość do wnioksku, iż
brakujące dane nie są w żaden sposób skorelowane z innymi zmiennymi.
Można stwierdzić, że braki w danych są losowe.

\subsubsection{Czyszczenie danych i imputacja brakujących
wartości}\label{czyszczenie-danych-i-imputacja-brakujux105cych-wartoux15bci}

Pierwszą rzeczą jest ustalenie reguł dla wszystkich danych istniejących
oraz imputowanych. Pozwala to na usunięcie wartości skrajnie
odstających, które mogły być błędem przy wprowadzaniu danych.

Po zastosowaniu zasad w jakich dane mają być skondensowane mogły
wystąpić wartości puste (NA). Wartości te powstały w miejscach wartości,
które nie spełniały powyższych zasad.

Po uprzednim przygotowaniu danych oraz ich zidentyfikowaniu należy
przejść do imputacji danych. Do imputacji wybrano metodę hotdeck, która
polega na zastępowaniu brakujących wartości rzeczywistych danymi z tego
samego zbioru.

Przed przystąpieniem do analizy konieczne było usunięcie brakujących
wartości oraz poprawienie błędów w danych.

Po zwizualizowaniu danych można zauważyć, że nie ma już pustych wartości
w danych. Zostały one skutecznie zastąpione za pomoca zastosowanej
metody.

\subsubsection{Analiza zależności między
zmiennymi}\label{analiza-zaleux17cnoux15bci-miux119dzy-zmiennymi}

Analiza korelacji i zależności między zmiennymi pozwala określić, które
czynniki mają największy wpływ na wyniki egzaminacyjne. Przeprowadzona
analiza obejmuje obliczenie współczynników korelacji Spearmana dla
kluczowych zmiennych oraz ich wpływu na wynik końcowy.

Przeprowadzona analiza wykazała, że frekwencja \emph{(Attendance)} oraz
liczba godzin nauki \emph{(Hours\_Studied)} mają najsilniejszy pozytywny
wpływ na wynik egzaminu. Istotna, choć słabsza korelacja występuje
również dla wcześniejszych wyników \emph{(Previous\_Scores)} oraz liczby
korepetycji \emph{(Tutoring\_Sessions)}.

Zaskakująco, zaangażowanie rodziców \emph{(Parental\_Involvement)} oraz
dostęp do zasobów edukacyjnych \emph{(Access\_to\_Resources)} wykazują
negatywną korelację, co może sugerować, że większa pomoc rodziców jest
wynikiem trudności ucznia, a samo posiadanie zasobów nie przekłada się
bezpośrednio na sukces.

Wyniki wskazują, że samodzielna praca i regularna obecność na zajęciach
są kluczowe dla osiągnięcia wysokich wyników egzaminacyjnych.

\subsubsection{Grupowanie danych i transformacja
zmiennych}\label{grupowanie-danych-i-transformacja-zmiennych}

Dane zostały podzielone na grupy w celu ułatwienia analizy i
interpretacji wyników.

Dla celów analizy dane zostały podzielone na grupy zmianna pokazująca
liczbę godzin nauki została podziolona na trzy grupy: poniżej 16, 16-23
oraz powyżej 23. Wyniki egzaminów zostały podzielone na 6 grup według
według progów akademickich, które dają zaliczenie egzaminu uczniom
którzy uzyskają wynik powyżej 60\%. Wyniki z poprzednich egzaminów
również zostały podzielone na 6 grup tak samo jak w przypadku wyników z
egzaminu końcowego.

\subsection{Eksploracyjna Analiza Danych
(EDA)}\label{eksploracyjna-analiza-danych-eda}

\subsubsection{Rozkład wyników
egzaminacyjnych}\label{rozkux142ad-wynikuxf3w-egzaminacyjnych}

Analiza rozkładu wyników egzaminacyjnych pozwala ocenić, jak
zróżnicowane są osiągnięcia uczniów oraz czy dane zawierają ewentualne
wartości odstające.

\begin{itemize}
\item
  Wyniki są skoncentrowane wokół średniej (67.23), z relatywnie małym
  rozrzutem (odchylenie standardowe 3.83).
\item
  Wysoka skośność i kurtoza wskazują na asymetryczność rozkładu i
  możliwe wartości odstające w wyższych wynikach (np. wyniki 90+).
\item
  Wyniki poniżej 60 oraz powyżej 90 mogą wymagać bliższego zbadania, aby
  określić, czy są to uczniowie ze szczególnymi trudnościami lub
  osiągający wyjątkowe wyniki.
\end{itemize}

\subsubsection{Analiza korelacji między
zmiennymi}\label{analiza-korelacji-miux119dzy-zmiennymi}

W celu oceny, które zmienne są najsilniej powiązane z wynikami
egzaminacyjnymi, oszcowano model regresji liniowej.

Model regresji liniowej przeprowadzony na dostępnych danych wykazał
istotne zależności między wieloma czynnikami a wynikami egzaminacyjnymi
uczniów. Model osiągnął R² = 0.6228, co oznacza, że około 62,3\%
wariancji wyników egzaminacyjnych można wyjaśnić uwzględnionymi
zmiennymi.

Najistotniejsze czynniki wpływające na wynik egzaminu

\paragraph{Kluczowe wnioski:}\label{kluczowe-wnioski}

\begin{enumerate}
\def\labelenumi{\arabic{enumi}.}
\item
  \textbf{Największy pozytywny wpływ na wyniki egzaminu mają:}

  \begin{itemize}
  \item
    Frekwencja (\texttt{Attendance}).
  \item
    Liczba godzin nauki (\texttt{Hours\_Studied}).
  \item
    Sesje korepetycji (\texttt{Tutoring\_Sessions}).
  \end{itemize}
\item
  \textbf{Negatywny wpływ na wyniki egzaminu mają:}

  \begin{itemize}
  \item
    Brak dostępu do zasobów edukacyjnych
    (\texttt{Access\_to\_Resources}).
  \item
    Niskie zaangażowanie rodziców (\texttt{Parental\_Involvement}).
  \item
    Niski poziom dochodów rodzinnych (\texttt{Family\_Income}).
  \end{itemize}
\item
  \textbf{Istotność zmiennych:} Wszystkie wymienione zmienne mają bardzo
  niski poziom wartości p (\texttt{p-value\ \textless{}\ 0.001}), co
  świadczy o ich wysokiej istotności statystycznej.
\end{enumerate}

\subsubsection{Wpływ poszczególnych czynników na wyniki
uczniów}\label{wpux142yw-poszczeguxf3lnych-czynnikuxf3w-na-wyniki-uczniuxf3w}

Przeanalizowano wpływ kluczowych czynników, takich jak liczba godzin
nauki, frekwencja, dostęp do zasobów edukacyjnych oraz wsparcie rodziców
na wyniki egzaminacyjne.

\textbf{Analiza wpływu godzin nauki na wynik egzaminu}

Po przeanalizowaniu powyższego wykresu można zauważyć, że uczniowie
którzy spędzają więcej czasu na nauce osiągają lepsze wyniki
egzaminacyjne.

Wpływ frekwencji na wynik egzaminu

Wpływ dostępu do zasobów edukacyjnych

Uczniowie z niskim dostępem do zasobów mają znacznie niższe wyniki
egzaminów w porównaniu do tych z wysokim dostępem. To jedna z kluczowych
zmiennych negatywnych.

Wpływ zaangażowania rodziców

Podobny wzorzec jak w przypadku zasobów --- niskie zaangażowanie
rodziców wiąże się z niższymi wynikami.

Wpływ dochodu rodziny na wynik egzaminu

Uczniowie z rodzin o niskim dochodzie osiągają niższe wyniki
egzaminacyjne w porównaniu do tych z rodzin o wyższym dochodzie.

Wpływ korepetycji

Większa liczba sesji korepetycji wpływa pozytywnie na wyniki
egzaminacyjne, choć efekt jest nieco słabszy.

\subsubsection{Podsumowanie projektu}\label{podsumowanie-projektu}

Przeprowadzona analiza wyników egzaminacyjnych dostarczyła cennych
informacji na temat czynników wpływających na sukces edukacyjny uczniów.
Skoncentrowano się na sześciu kluczowych zmiennych, które w sposób
istotny wpływają na wyniki: frekwencji, liczbie godzin nauki, liczbie
sesji korepetycji, dostępie do zasobów edukacyjnych, zaangażowaniu
rodziców oraz dochodzie rodziny.

Najważniejszymi czynnikami pozytywnie wpływającymi na wyniki egzaminów
okazały się frekwencja i liczba godzin nauki. Regularne uczęszczanie na
zajęcia oraz poświęcanie czasu na naukę stanowiły fundamenty wysokich
osiągnięć. Również sesje korepetycji w znacznym stopniu wspierały wyniki
uczniów, co sugeruje, że dodatkowe wsparcie edukacyjne może odgrywać
kluczową rolę w wyrównywaniu braków w wiedzy.

Zidentyfikowano również istotne bariery w edukacji. Ograniczony dostęp
do zasobów edukacyjnych, niski dochód rodziny oraz niskie zaangażowanie
rodziców wiązały się z gorszymi wynikami egzaminacyjnymi. Czynniki te są
szczególnie ważne w kontekście planowania polityk edukacyjnych i
programów wsparcia.

\subsubsection{Znaczenie analizy wyników
egzaminacyjnych}\label{znaczenie-analizy-wynikuxf3w-egzaminacyjnych}

Zrozumienie czynników wpływających na wyniki egzaminacyjne jest kluczowe
z kilku powodów:

\begin{enumerate}
\def\labelenumi{\arabic{enumi}.}
\item
  \textbf{Optymalizacja strategii nauczania} Wyniki analizy mogą być
  wykorzystane przez nauczycieli i decydentów edukacyjnych do
  dostosowania metod nauczania do indywidualnych potrzeb uczniów. Na
  przykład wsparcie uczniów z ograniczonym dostępem do zasobów
  edukacyjnych może obejmować udostępnienie technologii i materiałów
  dydaktycznych.
\item
  \textbf{Wsparcie dla uczniów w trudnej sytuacji} Identyfikacja
  kluczowych czynników, takich jak niski dochód rodziny czy brak
  zaangażowania rodziców, pozwala na wdrożenie programów wsparcia
  skierowanych do najbardziej potrzebujących uczniów. Działania takie
  mogą obejmować dodatkowe korepetycje, stypendia edukacyjne czy
  szkolenia dla rodziców.
\item
  \textbf{Poprawa efektywności edukacji} Lepsze zrozumienie procesów
  nauczania i czynników wpływających na wyniki pozwala na skuteczniejsze
  zarządzanie systemem edukacyjnym. Analiza wyników może służyć jako
  podstawa do projektowania polityk edukacyjnych, które koncentrują się
  na najważniejszych obszarach wymagających poprawy.
\end{enumerate}

Rekomendujemy dalsze inwestycje w programy wspierające uczniów z
trudnymi warunkami socjoekonomicznymi oraz promowanie aktywnego
zaangażowania rodziców w proces edukacji. Wyniki naszej analizy
podkreślają potrzebę holistycznego podejścia do edukacji, które
uwzględnia różnorodne czynniki wpływające na sukces uczniów. Tylko
poprzez odpowiednie wsparcie możemy zapewnić równe szanse wszystkim
uczniom i poprawić jakość edukacji jako całości.

\end{document}
